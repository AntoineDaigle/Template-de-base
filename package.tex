\usepackage[utf8]{inputenc}
\usepackage{supertabular}
\usepackage[T1]{fontenc}
\usepackage{icomma}
\usepackage{array} 
\usepackage{color}
\usepackage{amsmath,mathtools}
\usepackage{amssymb,amsfonts}
\usepackage{esint}
\usepackage{multirow}
\usepackage{float}
\usepackage{graphicx}
\usepackage{tikz}
\usepackage[left=2.5cm,right=2.5cm,top=3cm,bottom=3cm]{geometry}
\usepackage{hyperref}
\usepackage[french]{babel}
\usepackage{caption}
\usepackage[bottom]{footmisc}
\usepackage{gensymb}
\usepackage{tabulary}
\usepackage{fancyhdr}
\usepackage{siunitx}
\usepackage{textcomp}
\usepackage[siunitx]{circuitikz}
\newcommand{\HRule}{\rule{\linewidth}{0.5mm}}
\usepackage{parskip}
\setlength{\parindent}{20pt}
\setlength{\parskip}{15pt}
\setlength\extrarowheight{2pt}
\usepackage{tcolorbox}
\usepackage{enumitem}
\usepackage[ampersand]{easylist}
\usepackage{subfigure}
\usepackage{hhline}
\usepackage{datetime}
\usepackage{fancyhdr}
\pagestyle{fancy}
\usepackage{lastpage}
\renewcommand\headrulewidth{1pt}
\usepackage{biblatex} 
\addbibresource{Bibliography.bib} 
\usepackage{csquotes}
\usepackage{calligra}
\usepackage{physics}
\usepackage{listings}
\usepackage{xcolor}

\definecolor{codegreen}{rgb}{0,0.6,0}
\definecolor{codegray}{rgb}{0.5,0.5,0.5}
\definecolor{codepurple}{rgb}{0.58,0,0.82}
\definecolor{backcolour}{rgb}{0.95,0.95,0.92}

\lstdefinestyle{mystyle}{   %Un environement créer pour inclure du code dans le texte
    backgroundcolor=\color{backcolour},   
    commentstyle=\color{codegreen},
    keywordstyle=\color{magenta},
    numberstyle=\tiny\color{codegray},
    stringstyle=\color{codepurple},
    basicstyle=\ttfamily\footnotesize,
    breakatwhitespace=false,         
    breaklines=true,                 
    captionpos=b,                    
    keepspaces=true,                 
    numbers=left,                    
    numbersep=5pt,                  
    showspaces=false,                
    showstringspaces=false,
    showtabs=false,                  
    tabsize=2
}

\lstset{style=mystyle}

\DeclareMathAlphabet{\mathcalligra}{T1}{calligra}{m}{n} % "r" stylisé en électromagnétisme
\DeclareFontShape{T1}{calligra}{m}{n}{<->s*[2.2]callig15}{}
\newcommand{\scripty}[1]{\ensuremath{\mathcalligra{#1}}}
%INSCRIRE INFORMATION ICI!

\fancyhead[L]{\textbf{Titre}}
\fancyhead[C]{}
\fancyhead[R]{Antoine Daigle}
\renewcommand\footrulewidth{1pt}
\fancyfoot[C]{\textbf{Page \thepage/\pageref{LastPage}}}
\fancyfoot[R]{\today}